\documentclass[0-suturing.tex]{subfiles}
\begin{document}

\section{Needle Path Planning}
\noindent \textbf{Optimization Model}

The trajectory is discretized into time intervals $\mathcal{T} = \{0,1,\ldots,T\}$.


$\mathcal{T} \in [0, 1, \ldots, \mathcal{T}]$.
At each time step the needle's pose is parametrized as
$X_t = \begin{bmatrix}
    R_t & p_t \\ 0^T & 1
\end{bmatrix} \in SE(3)$
where $R_t \in SO(3)$ is a rotation matrix and
$ p_t \in  \mathbb{R}^3 $ is a position.

The control input is parametrized by the needle's radius $r_k$ and the distance $\Delta$, by which the needle is inserted at each time step. We note that $r_k$ is chosen from a discrete set available needle radii. And while $\Delta$ can be selected to be different for each time step, prior empirical results show that the optimization result in uneven distribution of $\Delta$ at each time step resulting undesirable trajectories.


\begin{align}
& \underset{\mathcal{X},\, r_k,\, \Delta}{\text{minimize}}
& & \alpha_\Delta \mathcal{C}_\Delta + \alpha_{Entry} C_{Entry} \\
& \text{subject to}
& & f_{collision}(\mathcal{X}_t, \mathcal{O}) > d_{safe}, & \forall t  \\
& & & curve(\mathcal{X}_{t+1}, \mathcal{X}_t) = r_k, & \forall t  \\
& & & \|p_i-p_1\|_2^2 < \epsilon \\
& & & \|p_f-p_T\|_2^2 < \epsilon \\
& & & T \Delta - 2 \pi l_{max}r_k - 2l_{gripper} < 0 
\end{align}

The costs and constraints in the above formulation are described below:

\subsubsection{Kinematic Constraints (Eq. (3))}
In order to minimize tissue damage, optimal needle trajectories should follow constant curvature paths through tissue. This condition is enforced by ensuring the curvature between timesteps along the trajectory remain 
constant. The constrained can be expressed expressed using Lie algebra defined on the SE(3) group described in the next section.
\begin{equation}
    X_{t+1} =  \text{exp}(\hat{\omega}) \cdot X_{t}
\end{equation}
Where $\omega = \begin{bmatrix} 0 & 0 & \Delta & \Delta r_k & 0 & 0\end{bmatrix}$.
This expression can be transformed into a standard non-convex equality constraint using the matrix log map:
\begin{equation}
    \log(X_{t+1} \cdot (\text{exp}(\hat{\omega}) \cdot X_{t})^{-1})^{\vee} = 0_6   
\end{equation}

\subsubsection{Collision Constraint (Eq. (2))}
We impose constraints to ensure that our trajectory avoids collision with a pre-defined avoidance volume.
This is a mesh can be constructed suture characteristics(width, depth) provided by a surgeon. Our technique
also allows this mesh to be general and non convex. This would allow mesh construction from visual inputs from sensory tools such as endoscopic cameras or ultrasound.  Hierarchical Approximate Convex Decomposition(HACD) \todo{cite} is used to decompose a given avoidance volume into a set of convex meshes 
$\mathcal{O} = \{O_1, \ldots, O_m\}$. We avoid collisions by ensuring that the signed distance between each 
point in the trajectory and each convex mesh in $\mathcal{O}$ is greater than a user defined safety margin $d_{safe}$. 


\subsubsection{Entry and Exit Point Constraints (Eq. (4) and (5))}
We define a maximum allowable error in $l_2$ distance $\epsilon$ for the start and end points of our trajectory.


\subsubsection{Needle Length Constraints (Eq. (6)) }
Our optimization needs to generate needle trajectories that are feasible within the physical dimensions of the needle. \todo{rephrase this}. 
The length of the trajectory can by approximated as $T\Delta$. In order to perform a successful needle insertion and pull through, the needle length needs to be longer than the trajectory length, and the two gripper widths. 
The maximum length of a needle is defined by the radius of curvature of the needle and the proportion of a full circle the needle's arc carves out. 
For most surgical applications needle arcs are either $\frac{3}{8}$, $\frac{1}{2}$, $\frac{5}{8}$ of a full circle. 
Thus we can define our max needle length as $2 \pi l_{max}r_k$ where $l_{max} = \frac{5}{8}$.
The full constraint then is reformulated in standard form in equation 6.


\subsubsection{Costs (Eq. (1))}
To penalize tissue damage we opimize for the smallest needle trajectory that satisfies our constraints.
We approximate the trajectory length as:
\begin{equation}
    C_{\Delta} = T\Delta 
\end{equation}
Surgical best practices suggest that needles should enter the tissue surface in a direction close to the surface normal.
We can use the lie algebra to express the constraint as the following:
\begin{equation}
    C_{Entry} = \|\log(X_f \cdot X_T^{-1})^{\vee}\|^2_2
\end{equation}


\subsection{\textbf{Re-parametrization of Pose Constraints}}
Generating collision-free, constant curvature trajectories in 3D space is challenging because it requires planning in the Special Euclidean Group in 3D, i.e. $SE(3)$ configuration space.
SE(3) is a semidirect product of Special Orthogonal Group (SO(3)) and $\mathbb{R}^3$. Using homogeneous coordinates, we can represent SE(3) as follows:
\begin{equation}
\begin{aligned}
SE(3) &= \left\{
\begin{bmatrix}
R & p \\
0 & 1
\end{bmatrix}
\in GL(4) \rvert R \in SO(3), t \in R^3 \right\}
\end{aligned}
\end{equation}

\noindent The action of an element $g \in SE(3)$ on a point $p \in \mathbb{R}^3$ is given by
$g = \begin{bmatrix}
R & p \\
0 & 1
\end{bmatrix},
\quad g \cdot p = Rp + t $.

% \begin{equation}
% \begin{aligned}
% g &=
% \begin{bmatrix}
% R & p \\
% 0 & 1
% \end{bmatrix} \\
% & g \cdot p = Rp + t
% \end{aligned}
% \end{equation}

Trajectory optimization methods that use sequential convex optimization, iteratively update a trajectory until the all the problem constraints are satisfied and objective function reaches a local minimum. However, optimization directly over the trajectory in pose representation (homogeneous coordinates in our case) can lead to poor results and slow convergence rates.

Rotation parametrizations such as quaternions or rotation matrices are over-constrained and can cause the optimization to be numerically unstable iterating over degenerate solutions.
While on the other hand, axis angle representations have the correct number of degrees of freedom but deformation at the poles of the configuration space severely slows down convergence.

Since $SE(3)$ is a differentiable manifold, we can instead optimize over the manifold using the corresponding Lie algebra, $\mathfrak{se}(3)$, which is defined as the tangent vector space at the identity of SE(3). An excellent introduction in lie algebraic approach in pose registration in provided by~\citet{Agrawal2006Lie}.

\noindent The Lie algebra of $SE(3)$ is given by
\begin{equation}
    \mathfrak{se}(3) = \left\{
    \begin{bmatrix}
        \hat{\omega} & u \\
        0 & 0
    \end{bmatrix}
    \in GL(4) \rvert \hat{\omega}\in so(3), u \in \mathbb{R}^3
    \right\}
\end{equation}

\noindent Here $\hat{\omega}$ is the skew symmetric form of the rotation vector
$\omega = (\omega_x, \omega_y, \omega_z)^T$ and is an element of the Lie algebra for $SO(3)$
\begin{equation}
    \hat{\omega} =
    \begin{bmatrix}
        0 & -\omega_z & \omega_y  \\
        \omega_z & 0 & -\omega_x \\
        -\omega_y & \omega_x & 0
    \end{bmatrix}
\end{equation}

\noindent The logarithm map $SE(3) \to \mathfrak{se}(3)$ is given by:
\begin{equation}
    \text{log}\left(
        \begin{bmatrix}
            R & t \\
            0 & 1
        \end{bmatrix}
    \right)  =
    \begin{bmatrix}
        log(R) & A^{-1}t \\
        0 & 0
    \end{bmatrix}
    \label{eq:logMap}
\end{equation}

\noindent where
\begin{equation*}
\begin{aligned}
    A^{-1} &= I - \frac{1}{2}\hat{\omega} +
\frac{2\sin\|\omega\| - \|\omega\|(1+cos\|\omega\|) }{2\omega^{2} \sin\|\omega\|}\hat{\omega}^{2},  \\
log(R) &= \frac{\phi}{2sin(\phi)}\big(R - R^T\big) \equiv \hat{\omega}
\end{aligned}
\end{equation*}
and $\phi$ satisfies $Tr(R) = 1+2cos(\phi), \quad |\phi|<\pi$.
\vspace{5pt}
\noindent The matrix exponential to map $\mathfrak{se}(3) \to SE(3)$ is given by
\begin{equation}
    \text{exp}\left(
        \begin{bmatrix}
            \hat{\omega} & u \\
            0 & 0
        \end{bmatrix}
    \right)  =
    \begin{bmatrix}
        \text{exp}(\hat{\omega}) & Au \\
        0 & 0
    \end{bmatrix}
    \label{eq:expMap}
\end{equation}

\noindent where $\text{exp}(\hat{\omega})$ can be computed with Rodrigue's formula:
\begin{equation*}
\begin{aligned}
    \text{exp}(\hat{\omega}) &= I + \frac{1-\cos(\|\omega\|)}{\|\omega\|}\hat{\omega}^2 +
    \frac{\sin(\|\omega\|)}{\|\omega\|^2}\hat{\omega},\\
    A &= I + \frac{1-\cos(\|\omega\|)}{\|\omega\|^2}\hat{\omega} + \frac{\|\omega\| - \sin\|\omega\|}{\|\omega\|^3}\hat{\omega}^2
\end{aligned}
\end{equation*}

With equations \eqref{eq:logMap} and \eqref{eq:expMap}, we can convert an element in SE(3) to $\mathfrak{se}(3)$ and vice-versa. Furthermore, we can also \todo{Jacobians}

\subsection{Modelling Uncertainty }
\todo{fill out the bsp formulation.}
Write about how to use the equations of the covariance matrix in the optimizaiton formulation using analytical derivatives.



\subsection{Suturing Best-Practice for Initialization}
\todo{rephrase and quantify}
There are many general rules that surgeons use to complete
a suture. A typical list of such rules is below [10] [11].
Manual needle sutures normally result in a picture similar to
Fig. 1.
1. The needle first “bites” the tissue orthogonally. By
inserting the needle such that the tip is orthogonal to
the tissue surface, tissue surface stress is minimized.
2. The wrench between the tissue and the needle during
the suture must be minimized. Minimizing the needle
tissue interaction force reduces the internal tissue stress,
and consequently reduces additional tissue trauma due
to the suture.
3. The re-grippable length of the needle during the
suture must be adequate for the needle re-grasp to be
completed successfully. Since the needle holder can
not be inserted through the tissue, there must be an
intermediate point during the suture that the gripper can
regrasp the needle on the tip to complete the suture.
4. The final depth of the needle in the tissue is an
important component of a successful suture. The actual
target depth is determined by many factors, including
both the wound being closed and the size of the needle.
5. The needle tip should only touch the tissue at the
insertion site. Similarly, the needle gripper should not
place unnecessary stress on the tissue.
The above list is not exhaustive, but details important components
of a quality suture.
Converting the listed suture principles into a list of analytic
equations allows for the planning algorithm to be automated
and optimized against the suture guidelines.

\end{document}
